    % specifies the documnt class. We usually use article but there are others. 
\documentclass[a4paper,12pt,oneside]{book}              

% these are standard packages used for the math symbols
\usepackage{amsmath,amssymb,amsthm, enumitem, hyperref, tabto} 
\usepackage[T1]{fontenc}
\usepackage[utf8]{inputenc}
\usepackage[english]{babel}
\usepackage{fancyhdr}
\usepackage{wrapfig}
\usepackage[fleqn]{amsmath}
\usepackage[utf8]{inputenc}
\usepackage{graphicx}
\usepackage{float}
\usepackage[absolute,overlay]{textpos}
\graphicspath{ {./Photos/} }
\usepackage{fancyhdr}
\usepackage[top=1in,bottom=1in,right=1in,left=1in]{geometry}
\usepackage{circuitikz}
\usepackage{tikz}
\usepackage{pgfplots}
\usetikzlibrary{decorations.markings,arrows}
\usetikzlibrary{datavisualization}
\usetikzlibrary{datavisualization.formats.functions}
\pgfplotsset{compat=newest}
\usepackage{amsmath}

% These commands below is to make sure the numbering of these are consistent with theorem
% If you are not sure what something means, delete them, build a new file and see the
% difference between the files. You can ignore this part for now.
\newtheorem{theorem}{Theorem}[section]
\newtheorem{conjecture}[theorem]{Conjecture}
\newtheorem{observation}[theorem]{Observation}
\newtheorem{definition}[theorem]{Definition}
\newtheorem{corollary}[theorem]{Corollary}
\newtheorem{lemma}[theorem]{Lemma}
\newtheorem{example}[theorem]{Example}
\newtheorem{remark}[theorem]{Remark}
\newtheorem{notation}[theorem]{Notation}


% Title of your project
\title{%
  \Huge Calculus \\
  \LARGE  Exploring \textbf{Real} Math
  }

% The author command places text right after title
\author{by \\
\Large The NUS High Math Interest Group (MIG) \\
}

\date{\Large 12th November 2022}

\begin{document}
\maketitle

\tableofcontents

\part{An Introduction}

\newpage
\chapter{Instantaneous Change}
\vspace{-30pt}
\large \textit{by Jiang Yuzhe}


\newpage
\chapter{How do Limits Work?}
\vspace{-30pt}
\large \textit{by Chong How}


\newpage
\chapter{Graphically Understanding the Rules}
\vspace{-30pt}
\large \textit{by Karimi Zayan}


\part{Physical Applications}

\newpage
\chapter{Movement}
\vspace{-30pt}
\large \textit{by Krishna}


\newpage
\chapter{Displacement}
\vspace{-30pt}
\large \textit{by Favian Lim}


\newpage
\chapter{Area under the Curve}
\vspace{-30pt}
\large \textit{by Ishan Sharma}


\part{Extensions}

\newpage
\chapter{Substituting}
\vspace{-30pt}
\large \textit{by Prannaya Gupta}

Differentiation and Integration often stems from a collective understanding of how small differentials change.




\end{document}
