    % specifies the documnt class. We usually use article but there are others. 
\documentclass[a4paper,12pt,oneside]{book}              

% these are standard packages used for the math symbols
\usepackage{amsmath,amssymb,amsthm, enumitem, hyperref, tabto} 
\usepackage[T1]{fontenc}
\usepackage[utf8]{inputenc}
\usepackage[english]{babel}
\usepackage{fancyhdr}
\usepackage{wrapfig}
\usepackage[fleqn]{amsmath}
\usepackage[utf8]{inputenc}
\usepackage{graphicx}
\usepackage{float}
\usepackage[absolute,overlay]{textpos}
\graphicspath{ {./Photos/} }
\usepackage{fancyhdr}
\usepackage[top=1in,bottom=1in,right=1in,left=1in]{geometry}
\usepackage{circuitikz}
\usepackage{tikz}
\usepackage{pgfplots}
\usetikzlibrary{decorations.markings,arrows}
\usetikzlibrary{datavisualization}
\usetikzlibrary{datavisualization.formats.functions}
\pgfplotsset{compat=newest}
\usepackage{amsmath}

% These commands below is to make sure the numbering of these are consistent with theorem
% If you are not sure what something means, delete them, build a new file and see the
% difference between the files. You can ignore this part for now.
\newtheorem{theorem}{Theorem}[section]
\newtheorem{conjecture}[theorem]{Conjecture}
\newtheorem{observation}[theorem]{Observation}
\newtheorem{definition}[theorem]{Definition}
\newtheorem{corollary}[theorem]{Corollary}
\newtheorem{lemma}[theorem]{Lemma}
\newtheorem{example}[theorem]{Example}
\newtheorem{remark}[theorem]{Remark}
\newtheorem{notation}[theorem]{Notation}


% Title of your project
\title{%
  \Huge Calculus \\
  \LARGE  Exploring \textbf{Real} Math
  }

% The author command places text right after title
\author{by \\
\Large The NUS High Math Interest Group (MIG) \\
}

\date{\Large 12th November 2022}

\begin{document}
\maketitle

\tableofcontents

\part{An Introduction}

\newpage
\chapter{Instantaneous Change}
\vspace{-30pt}
\large \textit{by Jiang Yuzhe}


\newpage
\chapter{How do Limits Work?}
\vspace{-30pt}
\large \textit{by Chong How}


\newpage
\chapter{Graphically Understanding the Rules}
\vspace{-30pt}
\large \textit{by Karimi Zayan}

\section{Power rule}

The power Rule states that

$$\frac{d}{dx}(x^n)=nx^{n-1}$$

 \noindent The symbolic proof is as follows
$$
\begin{aligned}
\frac{d}{dx}(x^n)&=\lim_{h\to 0}\frac{(x+h)^n-x^n}{h}\\
&=\lim_{h\to 0}\frac{x^n+\binom{n}{1}x^{n-1}h+\cdots +\binom{n}{n}h^n-x^n}{h}\\
&=\lim_{h\to 0}\frac{\binom{n}{1}x^{n-1}h+\cdots +\binom{n}{n}h^n}{h}\\
&=\lim_{h\to 0}\binom{n}{1}x^{n-1}+\cdots +\binom{n}{n}h^{n-1}\\
&=nx^{n-1}
\end{aligned}
$$

 \noindent To grasp an intuitive sense of this however, we can look at a geometric diagram for the equation $f(x)=x^2$.

\begin{figure}[H]
    \begin{center}
        \includegraphics[scale=0.35]{img/zayan/pr1.png}
        \caption{$f=x^2$}
        \label{fig:pr1}
    \end{center}
\end{figure}

 \noindent We look at $x^2$ and if we do a small nudge to $x$ by an infinitesimally small $dx$, we get the diagram as follows. We see that we can obtain an equation for $df$ by summing up the small rectangles that make up the expression.

$$df=xdx+xdx+{dx}^2$$

 \noindent Now, we solve for $\frac{df}{dx}$,
 
$$
\begin{aligned}
df&=xdx+xdx+{dx}^2\\
\frac{df}{dx}&=2x+dx\\
\frac{df}{dx}&=2x
\end{aligned}$$

\noindent We can apply this same logic to a cube.

\begin{figure}[H]
    \begin{center}
        \includegraphics[scale=0.75]{img/zayan/pr2.png}
        \caption{$f=x^3$}
        \label{fig:pr2}
    \end{center}
\end{figure}

\noindent We obtain a new equation for $df$ as follows

$$df = 3x^2dx+3x{dx}^2+{dx}^3$$

\noindent We can easily see

$$\frac{df}{dx}=3x^2$$

Overall, we see that for any $x^n$, nudging $x$ by $dx$ causes there to be $n$ sections with the value $x^{n-1}dx$ and a varying amount of other sections with areas proportional to $dx^2$ which can be ignored. Hence, we can see easily now why

$$\frac{d}{dx}(x^n)=nx^{n-1}$$

The power rule defined as such applies for not just integers but also all real numbers. However, the exact proof is beyond the scope of this lesson.

\part{Physical Applications}

\newpage
\chapter{Movement}
\vspace{-30pt}
\large \textit{by Krishna}


\newpage
\chapter{Displacement}
\vspace{-30pt}
\large \textit{by Favian Lim}


\newpage
\chapter{Area under the Curve}
\vspace{-30pt}
\large \textit{by Ishan Sharma}


\part{Extensions}

\newpage
\chapter{Substituting}
\vspace{-30pt}
\large \textit{by Prannaya Gupta}

Differentiation and Integration often stems from a collective understanding of how small differentials change.




\end{document}
